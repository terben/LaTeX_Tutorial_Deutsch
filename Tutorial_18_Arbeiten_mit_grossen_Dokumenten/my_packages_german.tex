% Paketdefinitionen für deutsche LaTeX Dokumente

% Pakete um deutsche Texte in LaTeX zu setzen:

% Wir setzen einen Text in neuer deutscher
% Rechtschreibung:
\usepackage[ngerman]{babel} 
% Eingabetext ist im utf-8 Format:
\usepackage[utf8]{inputenc} 
% Korrektes Ausgabeencoding für deutsche
% Umlaute:
\usepackage[T1]{fontenc} 
% Korrekte Anführungsstriche in deutschen Texten: 
\usepackage{csquotes} 

% Pakete für den Mathematikmodus:
\usepackage{amsmath}
\usepackage{amssymb}
% Zum Setzen von Einheiten
\usepackage{siunitx} 
% Wir wollen Zahlen/Einheiten nach
% deutschen Regeln setzen. siunitx nimmt
% leider nicht direkt die babel Spracheinstellung
\sisetup{locale = DE} 

% Figuren speichern wir immer in einem Unterordner
% 'figuren' unter unseren LaTeX Dokumenten
\usepackage{graphicx}
\graphicspath{{figuren/}}

% Das Paket biblatex ist für Literaturverweise
% zuständig.
\usepackage[backend=biber,   
            style=alphabetic % alphabetischer Zitierstil
           ]{biblatex}
           
% Meine Literaturdatenbank:
\addbibresource{literatur.bib}

% Sonstiges:

% Das Paket xspace sorgt für 'intelligente'
% Leerzeichnsetzung nach einem LaTeX Makro.
% siehe auch Datei my_newcommands_german.tex
\usepackage{xspace}

% Hin und wieder benötige ich ein wenig Dummy
% Text zum Vorführen wie hier in den Videos:
\usepackage{blindtext}
