% LaTeX Dokument zu Video 12 des YouTube LaTeX
% Tutorials
% (https://www.youtube.com/channel/UCgaFgieXi6HIryaFyhhzQtg)

\documentclass[12pt,a4paper]{scrartcl}

% LaTeX-Paket Definitionen
\usepackage[ngerman]{babel}
\usepackage[utf8]{inputenc}
\usepackage[T1]{fontenc}

\usepackage{amsmath}
\usepackage{amssymb}

% newcommand Definitionen:
%
% Abkürzungen für 'lange Zeichenketten'

% Der lange LaTeX-Befehl \textbackslash 
% erhält das Alias \tb.
\newcommand{\tb}{\textbackslash}

% Kommandos für den Mathematikmodus:

% Definition des Befehls \imag für
% eine korrekt gesetzte imaginäre Einheit
\newcommand{\imag}{\mathrm{i}}

% Definition des Befehls \euler für
% eine korrekte gesetzte eulersche Zahl
\newcommand{\euler}{\mathrm{e}}

% Der Befehl abs nimmt ein Argument das
% stets in korrekt grossen Absolutbetragstrichen
% gesetzt wird.
\newcommand{\abs}[1]{\left|#1\right|}

% Der Befeh deriv setzt eine Ableitung in Bruchform
% Erstes Argument gibt die abzuleitende Funktion, z.B.
% \sin(x) und das zweite Argument die Variable
% nach der abgeleitet wird.
\newcommand{\deriv}[2]{\frac{\mathrm{d} #1}{\mathrm{d} #2}}

%
\begin{document}
%
\section{\texttt{newcommand}-Beispiele}
%
%
\subsection{Demonstration des \texttt{\tb tb}-Befehls}
Meine Videos sind unter \texttt{C:\textbackslash
Benutzer\textbackslash Thomas\textbackslash Videos}. \\
Meine Videos sind unter \texttt{C:\tb Benutzer\tb Thomas\tb Videos}.
%
\subsection{Demonstration der \texttt{\tb euler} und
\texttt{\tb imag} Befehle}
\[
  \mathrm{e}^{\mathrm{i}\varphi}=\cos(\varphi)+\mathrm{i}\sin(\varphi)
  \quad
  \euler^{\imag\varphi}=\cos(\varphi)+\imag\sin(\varphi)
\]
%
\subsection{Demonstration des \texttt{\tb abs}-Befehls}
\[
  \left|a\right| \quad \left|\frac{a}{b}\right| \quad
  \abs{a} \quad \abs{\frac{a}{b}}
\]
%
\subsection{Demonstration des \texttt{\tb deriv}-Befehls}
\[
  \frac{\mathrm{d}\sin(x)}{\mathrm{d}x}=\cos(x) \quad
  \deriv{\sin(x)}{x}=\cos(x)\quad
  \deriv{}{x}\sin(x)=\cos(x)
\]
\end{document}
