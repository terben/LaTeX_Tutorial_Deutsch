% Erweiterte Fassung des Vorführdokuments aus Video 12
% des LaTeX Tutorials auf Youtube
% (https://www.youtube.com/channel/UCgaFgieXi6HIryaFyhhzQtg)
%
\documentclass[11pt,a4paper]{scrartcl}

\usepackage[ngerman]{babel}
\usepackage[utf8]{inputenc}
\usepackage[T1]{fontenc}
\usepackage{csquotes}

\usepackage{amsmath}
\usepackage{amssymb}

\usepackage{siunitx}  % Zum Setzen von Einheiten
\sisetup{locale = DE} % Wir wollen Zahen/Einheiten nach
                      % deutschen Regeln setzen.
                      % siunitx nimmt leider nicht
                      % direkt die babel Spracheinstellung

\begin{document}
%
\section{Typische Fehler beim Formelsatz}
Bitte beachten Sie in der folgenden Formel:
%
\begin{itemize}
  \item Mathematische Gleichungen und Objekte im Fließtext
    \emph{immer} im Mathematikmodus (in eingeschlossenen
    Dollarzeichen) setzen!
  \item Verwenden Sie für Funktionsnamen im Mathematikmodus
    die vorhandenen \LaTeX{} Befehle anstatt sie \emph{von Hand}
    zu setzen!
  \item Setzen Sie Text innerhalb des Mathematikmodus mit dem
    \LaTeX{}-Befehl \texttt{\textbackslash text}!    
\end{itemize}
\textbf{Falscher Formelsatz:} Mit der Variable x gilt: 
\[
  sin(x)=0   \quad mit \quad  x=n\pi, n\in\mathbb{Z}.
\]
\textbf{Korrekter Formelsatz:} Mit der Variable $x$ gilt: 
\[
  \sin(x)=0   \text{  mit  }  x= n\pi, n\in\mathbb{Z}.
\]
Bitte beachten Sie in der folgenden Formel, dass die imaginäre Einheit
\enquote{$\mathrm{i}$} und die eulersche Zahl \enquote{$\mathrm{e}$}
aufrecht mit dem Befehl \texttt{\textbackslash mathrm} zu setzen sind:\\
\textbf{Falscher Formelsatz:}
\[
  e^{i\varphi}=\cos(\varphi)+i\sin(\varphi)
\]
\textbf{Korrekter Formelsatz:} 
\[
  \mathrm{e}^{\mathrm{i}\varphi}=\cos(\varphi)+\mathrm{i}\sin(\varphi)
\]
Bitte beachten Sie in der folgenden Formel, dass der
Differentialoperator \enquote{$\mathrm{d}$} aufrecht mit dem
Befehl \texttt{\textbackslash mathrm} zu setzen ist:\\
\textbf{Falscher Formelsatz:}
\[
  \left(\frac{d^{2}}{dr^{2}} + \frac{1}{r}\frac{d}{dr}\right)\psi(r)=h(r)
\]
\textbf{Korrekter Formelsatz:} 
\[
  \left(\frac{\mathrm{d}^{2}}{\mathrm{d}r^{2}} +
  \frac{1}{r}\frac{\mathrm{d}}{\mathrm{d}r}\right)\psi(r)=
  h(r)
\]
%
\section{Zahlen und Einheiten im Mathematikmodus}
%
Verwenden Sie für \emph{alles} was mit physikalischen Zahlen und
Einheiten zu tun hat das Paket \texttt{siunitx}. Im Folgenden sind die
ersten drei Beispiele \emph{von Hand}, und verkehrt, gesetzt! Die
Fehler sind: (1) Eine Zahl $3.0e03$ wurde direkt aus der Ausgabe eines
Computerprogramms entnommen. Sie entspricht in keinster Weise Vorgaben
für Zahlendarstellung in der Literatur; (2) Einheiten werden immer
\emph{aufrecht} gesetzt; (3) Der Dezimaltrenner im Deutschen ist ein
Komma und kein Punkt; (3) Zwischen Zahl und Abstand muss ein kleiner
Abstand stehen. \texttt{siunitx} kümmert sich \emph{automatisch} um
all diese Probleme!
%
\begin{center}
  \begin{tabular}{ll}
    $l=3.0e03 km$ &  Zahl aus Programm; Einheit nicht aufrecht \\
    $l=3.0\cdot 10^{3} km$ & Einheit nicht aufrecht \\
    $l=3.0\cdot 10^{3} \mathrm{km}$ & korrekt?\\
    $l=\SI{3.0e03}{\kilo\meter}$ & richtig! \\
  \end{tabular}
\end{center}

In der ersten Version des folgenden Satzes sind Groß- und
Kleinschreibweisen von Einheiten verkehrt (kilo=k, Mega=M, Hertz=Hz): \\
Die Frequenz betrug nur $f=1.0 \mathrm{khz}$ anstatt der erwarteten
$f=1.0 \mathrm{mhz}$. \\
Die Frequenz betrug nur $f=\SI{1.0}{\kilo\hertz}$ anstatt der
erwarteten $f=\SI{1,0}{\mega\hertz}$. \\
Auch um diese Probleme kümmert sich \texttt{siunitx} automatisch!
%
\end{document}
