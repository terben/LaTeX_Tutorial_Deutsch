% Kommentierter Quelltext zu Video 15 des LaTeX Tutorials
% von Thomas Erben
%
\documentclass[a4paper, 12pt]{scrartcl}

% Hier werden einige Pakete eingebunden
\usepackage[utf8]{inputenc}               % Direkte Eingabe von ä usw.
\usepackage[T1]{fontenc}                  % Font Kodierung für die Ausgabe
\usepackage[ngerman]{babel}               % Verschiedenste
                                          % sprach-spezifische Extras
\usepackage[autostyle=true]{csquotes}     % Intelligente Anführungszeichen

\title{Der Hobbit}
\author{John R.\,R.~Tolkien}
\date{London, 1937}

\begin{document}
%
%\maketitle
%
\section*{Vorwort}
Das Datum ist das der englischen Erstausgabe. Der Text kommt natürlich
aus der deutschen Übersetzung (\enquote{Der kleine Hobbit} erschienen
1999 im dtv Verlag).
%
\section{Eine unvorhergesehene Gesellschaft}
%
\subsection*{Die Hobbithöhle}
In einer Höhle in der Erde, da lebte ein \textbf{Hobbit}. 
Nicht in einem schmutzigen, nassen Loch, 
in das die Enden von irgendwelchen Würmern herabbaumelten 
und das nach Schlamm und Moder roch. 
Auch nicht etwa in einer trockenen Kieshöhle, die so kahl war, 
dass man sich nicht einmal niedersetzen oder gemütlich frühstücken konnte. 
Es war eine \underline{Hobbithöhle}, und das bedeutet Behaglichkeit.
%
\subsection*{Ein guter Morgen (?)}
Alles, was also der keineswegs misstrauische Bilbo an diesem Morgen sah,
war ein kleiner, alter Mann mit einem Stab, hohem, spitzem blauen Hut, 
einem langen, grauen Mantel, mit einer silbernen Schärpe,  
über die sein langer, silberner Bart hing, 
ein kleiner, alter Mann mit riesigen schwarzen Schuhen.
%
\enquote{Guten Morgen}, sagte Bilbo, und er meinte es ehrlich. 
Die Sonne schien, und das Gras war grün. 
Aber Gandalf schaute ihn scharf unter seinen buschigen Augenbrauen hervor an.

\enquote{Was meint Ihr damit?} fragte er. 
\begin{itemize}
  \item \enquote{\emph{Wünscht Ihr mir einen guten Morgen?}}

  \item \enquote{\emph{Oder meint Ihr, dass dies ein guter Morgen
        ist, gleichviel, ob ich es wünsche oder nicht?}}
  \item \enquote{\emph{Meint Ihr, dass Euch der Morgen gut bekommt?}}
  \item \enquote{\emph{Oder dass dies ein Morgen ist, an dem man gut
        sein muss?}}
\end{itemize}

\enquote{Alles auf einmal}, sagte Bilbo. \enquote{Wie heißt Ihr
  eigentlich?} fragte der Hobbit. \enquote{Ich bin Gandalf, und Gandalf,
denkt nur, das bin ich!} antwortete der Zauberer.
%
\section{Die Zwerge}
Mit dem Namen Gandalf fiel bei Bilbo der Groschen. Der Zauberer war
früher oft zu Gast bei den Hobbits. Er hatte seinem Grossvater vor
Ewigkeiten ein paar magische Diamantenklammern geschenkt und die
Hobbits zur Sommersonnenwende stets mit beeindruckenden Feuerwerken
erfreut. Nach einer We
ile lud Bilbo den Zauberer für den nächsten Tag
zum Tee ein und dieser verschwand so schnell wie er gekommen war.
 
Bevor Gandalf am folgenden Nachmittag erschien, wurde der arme Hobbit
von 13 ungebetenen Gästen, es waren Zwerge, heimgesucht. Ihre Namen
finden sich, zusammen mit einigen Zusatzinformationen, in
Tabelle~\ref{tab:zwerge}.
% 
% Die folgende Tabelle ist als Gleitobjekt realisiert. LaTeX bekommt
% die Freiheit sie an eine günstige Stelle des Ausgabedokuments zu
% platzieren.
% Ein Tabellen Gleitobjekt hat folgende Elemente:
%
% - De eigentliche tabular Umgebung ist in eine table Umgebung
%   eingebettet.
% - Das Zentrieren der Tabelle geschieht durch einen \centering
%   Befehl anstatt durch eine center Umgebung.
% - Als Gleitobjekt hat die Tabelle eine Über- bzw. Unterschrift
% - Wir benötigen auch ein label um die Tabelle im Text zu referenzieren.
\begin{table}
  \centering
  % Die folgenden zwei Befehle erzeugen eine Tabellenüberschrift
  \captionabove{Die dreizehn Zwerge}
  \label{tab:zwerge}
  \begin{tabular}{l||l|l|l|l|l}
  Name   & Bart & Gürtel & Kapuze & Instrument & Sonstiges \\
  \hline
  Dwalin & blau & gold & dunkelgrün & Bratsche & \\
  Balin & weiß & & purpurrot & Bratsche & \\
  Kili & gelb & silber & blau & Fiedel & Werkzeug \\
  Fili & gelb & silber & blau & Fiedel & Spaten \\
  Dori & & gold & purpurrot & Flöte & \\
  Nori & & gold & purpurrot & Flöte & \\
  Ori & & gold & grau & Flöte & \\
  Oin & & silber & braun & & \\
  Gloin & & silber & silber & & \\
  Bifur & & & gelb & Klarinette & \\
  Bofur & & & gelb & Klarinette & \\
  Bombur & & & blaßgrün & Trommel & fett \\
  Thorin & & & himmelblau mit & Harfe & sehr berühmt \\
         & & & silberner Schärpe & &
  \end{tabular}
  % Falls Sie eine Tabellenunterschrift bevorzugen kommentieren
  % aktivieren Sie die zwei folgenden Zeilen und kommentieren Sie
  % die entsprechenden Zeilen oben für die Überschrift aus:
  %\caption{Die dreizehn Zwerge}
  %\label{tab:zwerge}
\end{table}
%
\section{Gebratenes Hammelfleisch}
Als Bilbo am Morgen aufsteht, glaubt er, dass die Zwerge ohne ihn
fortgegangen sind und war enttäuscht. Doch Gandalf klärt ihn auf.
Bilbo beeilt sich, damit er noch rechtzeitig zum Treffpunkt kommen
kann, wo die Zwerge bereits auf ihn warten.Bilbo, Gandalf und die
Zwerge machen sich auf die Reise.Eines Abends machen sie eine Pause
und bemerken, dass Gandalf verschwunden ist. In einem Wald entdecken
sie ein Licht und die Zwerge schicken Bilbo, um es auszukundschaften.
An der Feuerstelle, von der das Licht kommt, sitzen drei Trolle, die
gerade dabei sind, Hammelfleisch zu braten. Als Bilbo versucht, die
Geldbörse eines Trolls zu stehlen, wird er entdeckt und auch die
Zwerge werden gefangen genommen. Doch bevor die Trolle sie verspeisen
können, werden sie von Gandalf überlistet und vom Sonnenlicht bei
Tagesanbruch in Steine verwandelt. Schließlich entdecken die Gefährten
die Höhle der Trolle und nehmen alle Goldmünzen und Lebensmittel mit,
die sie dort fanden.
\end{document}
