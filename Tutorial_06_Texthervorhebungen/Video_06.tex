% Eine dokumentierte Version der LaTeX Datei von
% Teil sechs des LaTeX Video Tutorials
% (siehe https://youtu.be/Eyo8cPaENLc)
\documentclass[a4paper,12pt]{scrartcl}

\usepackage[ngerman]{babel}   % Deutsche Einstellungen
\usepackage[utf8]{inputenc}   % utf-8 Eingabe
\usepackage[T1]{fontenc}      % Korrekte Ausgabefonts im
                              % Ausgabedokument
\usepackage{csquotes}         % Setzen von Anführungsstrichen

\begin{document}
%
\section{Texthervorhebungen}
% Texthervorhebungen oder Änderungen von Textfonts werden in LaTeX
% durch LaTeX-Befehle realisiert. Texte, die hervorgehoben werden
% sollen, werden innerhalb geschweifter Klammern nach einem
% entsprechenden Befehl geschrieben:
%
Windows und Linux sind gute Betriebssysteme. \\
% Der Befehl '\emph' hebt Text hervor. "Wie" Text in einem gegebenem
% Kontext hervorgehoben wird, wird von LaTeX selber bestimmt. Hier wird
% der Text kursiv geschrieben.
%
Windows und Linux sind \emph{gute} Betriebssysteme.  \\
% Der Befehl '\textbf' druckt Text fett.
%
Windows und Linux sind \textbf{gute} Betriebssysteme.  \\
%
% \texttt setzt Text in nicht-proportionaler Schreibmaschinenschrift.
% LaTeX-Befehle können auch ineinander geschachtelt werden um ihren
% Effekt auf Text zu kombinieren:
\texttt{Windows} und \texttt{Linux} sind \emph{\textbf{gute}}
Betriebssysteme.  \\
%
\section{Anführungsstriche}
% Texte in wörtlicher Rede werden in LaTeX mit dem Befehl '\enquote'
% gesetzt (zweiter Satz). Vermeiden Sie das Setzen von wörtlicher Rede
% mit einfachen 'Hochkommas'! Dies ist schlichtweg falsch!
%
Thomas sagte "Guten Morgen". \\        % So bitte nicht!
Thomas sagte \enquote{Guten Morgen}.   % So ist es richtig!
\end{document}
