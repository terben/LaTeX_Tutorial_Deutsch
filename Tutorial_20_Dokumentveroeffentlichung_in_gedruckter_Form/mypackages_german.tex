% Paketdefinitionen für deutsche LaTeX Dokumente

% Pakete um deutsche Texte in LaTeX zu setzen:
\usepackage[ngerman]{babel} % Wir setzen einen deutschen Text
                            % in neuer deutscher Rechtschreibung
\usepackage[utf8]{inputenc} % Eingabetext ist im UTF-8 Format)
                            % kodiert
\usepackage[T1]{fontenc} % Korrektes Ausgabeencoding
                         % für deutsche Sonderzeichen 
\usepackage{csquotes} % korrekte Anführungsstricke in
                      % deutschen Texten

% Pakete für den Mathematikmodus:
\usepackage{amsmath}
\usepackage{amssymb}
\usepackage{siunitx}  % Zum Setzen von Einheiten
\sisetup{locale = DE} % Wir wollen Zahen/Einheiten nach
                      % deutschen Regeln setzen.
                      % siunitx nimmt leider nicht
                      % direkt die babel Spracheinstellung

% Figuren speichern wir immer in einem Unterordner
% 'figuren' unter unseren LaTeX Dokumenten
\usepackage{graphicx}
\graphicspath{{figuren/}}

% Das Paket biblatex ist für Literaturverweise zuständig.
\usepackage[backend=biber,   
            style=alphabetic % alphabetischer Zitierstil
           ]{biblatex}

% Literaturdatenbank:
\addbibresource{literatur.bib}

% Sonstiges:

% Das Paket xspace sorgt für 'intelligente' Leerzeichnsetzung
% nach einem LaTeX Makro - siehe auch Datei my_newcommands_german.tex
\usepackage{xspace}
