% Quelltext zu 'mathe_umgebungen.pdf'
% Begleitmaterial zu Tutorial 10 des YouTube Video Tutorials (siehe
% https://www.youtube.com/channel/UCgaFgieXi6HIryaFyhhzQtg) von Thomas
% Erben
%
\documentclass[a4paper,12pt]{article}

\usepackage[ngerman]{babel}  % Deutsche Spracheinstellungen
\usepackage[utf8]{inputenc}  % utf-8 Eingabeencoding 
\usepackage[T1]{fontenc}     % Ausgabeencoding für Sonderzeichen
\usepackage{csquotes}        % Für Anführungszeichen (Gänsefüßchen)

% Die folgenden drei Pakete möchte man für Mathematik eigentlich
% immer!
\usepackage{amsmath}
\usepackage{amssymb}
\usepackage{amsfonts}

\title{Demonstration Mathematischer Umgebungen}
\author{Thomas Erben}
\date{\today}
%
\begin{document}
\maketitle

%
% Der folgende Text demonstriert die verschiedenen mathematischen Modi.
% - $ ... $ fuer Formeln innerhalb des laufenden Textes
% - \[ ... \] fuer vom Text abgesetzte Formeln
% - \begin{equation} ... \end{equation} fuer vom Text abgesetzte Formeln
%   mit fortlaufender Numerierung
% - \begin{align} ... \end{align} fuer vom Text abgesetzte, mehrzeilige
%   Formeln mit fortlaufender Numerierung. Die einzelnen Zeilen sehen so aus:
%
%   linker Formelteil 1 & = rechter Formelteil 1 \\
%   linker Formelteil 2 & = rechter Formelteil 2 \nonumber \\
%
%   Das '&' steht hierbei vor dem Zeichen nach dem die Formeln
%   ausgerichtet werden. Im Normalfall ist dies ein Leerzeichen.
%   Ein '\nonumber' in einer der Zeilen heisst, dass diese Zeile nicht
%   numeriert wird!
\section{Die Summenformel von Gauss}
Die Summe der Zahlen $1\ldots n$ berechnen wir mit einer bekannten Formel
von Gauss mit $1+2+3+\ldots+n=\frac{n(n+1)}2$. Vom Text abgesetzt sieht 
das Ganze so aus:
%
\[
  1+2+3+\ldots+n=\frac{n(n+1)}2.
\]
Natürlich können wir Gleichungen auch Nummern geben:
%
\begin{equation}
  1+2+3+\ldots+n=\frac{n(n+1)}2.
\end{equation}
%
\section{Darstellungen der Eulerschen Zahl
  $\mathrm{e}$}
Diese Gleichungen werden dann automatisch fortlaufend numeriert, z. B. wenn wir
die Eulersche Zahl $\mathrm{e}$ mit einer unendlichen Summe berechnen:
%
\begin{equation}
  \mathrm{e}=\sum_{n=1}^{\infty}\frac 1{n!}.
\end{equation}
Eine andere Darstellung dieser Zahl ist:
%
\begin{equation}
  \mathrm{e}=\lim_{n\to\infty}\left(1+\frac 1n\right)^n.
\end{equation}
Man beachte hier dass die Eulersche Zahl $\mathrm{e}$ in
mathematischen Formeln \emph{aufrecht} gesetzt wird. Dies erreicht man
im Quelltext mit \enquote{\texttt{\textbackslash mathrm\{e\}}}.
%
\section{Die Binomischen Formeln}
Die drei binomischen Formeln lauten:
%
\begin{align}
  (a+b)^2 & = a^2+2ab+b^2 \nonumber \\
  (a-b)^2 & = a^2-2ab+b^2  \\
  (a+b)\cdot(a-b) & = a^2-b^2 \nonumber
\end{align}
%
\end{document}
